% Definē dokumenta tipu
\documentclass[12pt]{report}

% Importējam vajadzīgās pakotnes
\usepackage{polyglossia} % Pakotne valodu lietošanai
\usepackage{graphicx} % Pakotne attēlu lietošanai
\setdefaultlanguage{latvian} % Iestatām noklusējum valodu

\begin{document}
\chapter{Lielā nodaļa}
\section{Apakšnodaļa}
\subsection{Apakšnodaļa apakšnodaļai}


Testējam latviešu simbolus.

\textit{slīpraksts}, 
\textbf{treknraksts}, 
\underline{pasvītrojums}.


% Jauna rindkopa = 2 tukšas līnijas
Šim vajadzētu būt jaunā rindkopā

% Nenumurēts saraksts
\begin{itemize}
  \item Āboli
  \item Burkāni
  \item Sīpoli
\end{itemize}
% Sanumurēts sarakts
\begin{enumerate}
  \item Janvāris
  \item Februāris
\end{enumerate}

\begin{equation} 
  \frac{x}{y}
  \right
\end{equation}

Nenumurēta formula jaunā rindā: $$c^2=a^2+b^2$$.
Nenumurēta formula tajā pat rindā: $c^2=a^2+b^2$.

% Attēlu ievietošana
\begin{figure}[!ht] % [!ht] - liek bildi tieši tur, kur beidzas rindkopa
  \centering
  \includegraphics[width=0.8\textwidth]{https://latexbase.com/images/raptor.jpg}
  \caption{Velociraptor no LaTeXBase}
  \label{fig:velociraptor}
\end{figure}

Attēlā Nr. \ref{fig:velociraptor} redzams Velociraptors.

% Tabulu veidošana LaTeX
\begin{table}[!ht]
  \begin{tabular}{|l|l}
    % |, lai būtu iezīmēts režģis.
    % l, lai teksts būtu pa kreisi
    \hline
    Pārtikas produkts & Daudzums (g) \\ % Atdala šūnas ar &. Rindiņu beidz ar \\
    \hline
    Šokolāde & 250 \\ \hline
    Sviests & 250 \\ \hline
    Milti & 50 \\ \hline
    Cukurs & 250 \\ \hline
  \end{tabular}
  \caption{Šokolādes kūkas sastāvdaļas}
  \label{tab:recepte}
\end{table}

Tabulā Nr. \ref{tab:recepte} ir redzamas kūkas sastāvdaļas.

\end{document}